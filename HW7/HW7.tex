% --------------------------------------------------------------
% This is all preamble stuff that you don't have to worry about.
% Head down to where it says "Start here"
% --------------------------------------------------------------
 
\documentclass[12pt]{article}
 
\usepackage[margin=1in]{geometry} 
\usepackage{amsmath,amsthm,amssymb}
\usepackage{braket}
\usepackage{graphicx}
\usepackage{calligra}
\usepackage{calrsfs}
\usepackage{subcaption}
\usepackage{listings}
\newcommand{\N}{\mathbb{N}}
\newcommand{\Z}{\mathbb{Z}}
 
\newenvironment{theorem}[2][Theorem]{\begin{trivlist}
\item[\hskip \labelsep {\bfseries #1}\hskip \labelsep {\bfseries #2.}]}{\end{trivlist}}
\newenvironment{lemma}[2][Lemma]{\begin{trivlist}
\item[\hskip \labelsep {\bfseries #1}\hskip \labelsep {\bfseries #2.}]}{\end{trivlist}}
\newenvironment{exercise}[2][Exercise]{\begin{trivlist}
\item[\hskip \labelsep {\bfseries #1}\hskip \labelsep {\bfseries #2.}]}{\end{trivlist}}
\newenvironment{reflection}[2][Reflection]{\begin{trivlist}
\item[\hskip \labelsep {\bfseries #1}\hskip \labelsep {\bfseries #2.}]}{\end{trivlist}}
\newenvironment{proposition}[2][Proposition]{\begin{trivlist}
\item[\hskip \labelsep {\bfseries #1}\hskip \labelsep {\bfseries #2.}]}{\end{trivlist}}
\newenvironment{corollary}[2][Corollary]{\begin{trivlist}
\item[\hskip \labelsep {\bfseries #1}\hskip \labelsep {\bfseries #2.}]}{\end{trivlist}}
 
\begin{document}
 
% --------------------------------------------------------------
%                         Start here
% --------------------------------------------------------------
 
%\renewcommand{\qedsymbol}{\filledbox}
 
\title{HW7}
\author{Carl Mueller\\ %replace with your name
CSCI 5254 - Convex Optimization} %if necessary, replace with your course title
\maketitle

\subsection*{8.16}
Formulate the following asa CVX optimization problem:\\
Find the rectangle
$$R = \set{x \in \textbf{R}^n | l \preceq x \preceq u}$$
of maximum volume enclosed in the polyhedron
$$P = \set{x | Ax \preceq b}$$\\
The volume can be expressed as:
\begin{proposition}{1}
\begin{align}
v = \prod_{i=1}^{n}u_i-l_i
\end{align}
\end{proposition}
We want all the $2^n$ corners do be contained within the polyhedron. This every corner must meet the polyhedron constraint $Ac \preceq b$. Where $c$ is the vector of corners. Each of these corners can be be more succinctly represented as the vector based on the upper and lower values of each edge:\\ 
If we express $x_i$ as $u_i-l_i$ then this system becomes
$$\sum_{i=1}^{n}a_{ij}(u_j-l_j) \le b_i$$\\
The problem can be expressed as:
\begin{equation*}
\begin{aligned}
& \underset{}{\text{minimize}}
& & \prod_{i=1}^{n}u_i-l_i\\
& \text{subject to}\
& &\sum_{i=1}^{n}a_{ij}(u_j-l_j) \le b_i
\end{aligned}
\end{equation*}
The constraint is a posynomial as it is a summation of the monomial $a_{ij}(u_j-l_j)$.\\
To make the problem a non-linear geometric optimization problem, we take the log of the objective:
\begin{equation*}
\begin{aligned}
& \underset{}{\text{minimize}}
& & \sum_{i=1}^{n}log(u_i-l_i)\\
& \text{subject to}\
& &\sum_{i=1}^{n}a_{ij}(u_j-l_j) \le b_i
\end{aligned}
\end{equation*}

\subsection*{8.24}
We make use of the Cauchy-Schwarz inequality and sustite $p$ knowing that $||u||_2 \le p$:
\begin{proposition}{1}
\begin{align}
u^tx_i \le ||u||_2|x_i||_2\\
||u||_2||x_i||_2 \le p||x_i||_2\\
u^ty_j \le ||u||_2|y_j||_2\\
||u||_2|y_j||_2 \le p||y_j||_2\\
-||u||_2||y_j||_2 \ge -p||y_j||_2\\
\end{align}
\end{proposition}
For $x_i$:
\begin{equation*}
\begin{aligned}
(a+u)^Tx_i \ge b\\
a^Tx_i+u^Tx_i \ge b\\
a^Tx_i+||u||_2|x_i||_2 \ge b\\
a^Tx_i+p|x_i||_2 \ge b\\
a^Tx_i-b \ge -p|x_i||_2
\end{aligned}
\end{equation*}
For $x_i$:
\begin{equation*}
\begin{aligned}
(a+u)^Ty_j \le b\\
a^Ty_j+u^Ty_j \le b\\
a^Ty_j+||u||_2||y_j||_2 \le b\\
a^Ty_j+p||x_i||_2 \le b\\
a^Ty_j-b \le -p||y_j||_2\\
b-a^Ty_j \ge p||y_j||_2
\end{aligned}
\end{equation*}
The optimization problem:
\begin{equation*}
\begin{aligned}
& \underset{}{\text{minimize}}
& & p\\
& \text{subject to}\
& & b-a^Ty_j \ge p||y_j||_2\\
&&& a^Tx_i-b \ge -p|x_i||_2\\
&&& ||a||_2 \le 1
\end{aligned}
\end{equation*}\\
\textbf{Additional Exercises:}\\
\subsection*{5.12}
One heurisitc is to treat P as noise.

\subsection*{5.18}
We can reformulate the problem as the original object being less than or equal to some value $z$:
\begin{equation*}
\begin{aligned}
1+max_{k\neq y_i} f_k(x_i)  - f_{y_i} (x_i) \leq z_i,  z_i\geq 0\\
\end{aligned}
\end{equation*}\\
This can be represented by the following problem:
\begin{equation*}
\begin{aligned}
& \underset{}{\text{minimize}}
& &  \sum_i z_i + \mu ||A||_F^2\\
& \text{subject to}\
& & 1+max_{k\neq y_i} f_k(x_i) - f_{y_i} (x_i) \leq z_i, \forall i\\
&&&  1^T b= 0,   z \geq 0\\
\end{aligned}
\end{equation*}\\
This can be reexpressed using the individual inequality constraints:
\begin{equation*}
\begin{aligned}
& \underset{}{\text{minimize}}
& &  \sum_i z_i + \mu ||A||_F^2\\
& \text{subject to}\
& &  1^T b= 0,   z \geq 0\\
&&&  1+a_k^Tx_i + b  - y_i \leq z_i,    k = 1, 2, …,  y_i -1, y_i +1, …, K, i=1, 2, …, m
\end{aligned}
\end{equation*}\\

% --------------------------------------------------------------
%     You don't have to mess with anything below this line.
% --------------------------------------------------------------
 
\end{document}