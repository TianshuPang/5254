% --------------------------------------------------------------
% This is all preamble stuff that you don't have to worry about.
% Head down to where it says "Start here"
% --------------------------------------------------------------
 
\documentclass[12pt]{article}
 
\usepackage[margin=1in]{geometry} 
\usepackage{amsmath,amsthm,amssymb}
\usepackage{braket}
 
\newcommand{\N}{\mathbb{N}}
\newcommand{\Z}{\mathbb{Z}}
 
\newenvironment{theorem}[2][Theorem]{\begin{trivlist}
\item[\hskip \labelsep {\bfseries #1}\hskip \labelsep {\bfseries #2.}]}{\end{trivlist}}
\newenvironment{lemma}[2][Lemma]{\begin{trivlist}
\item[\hskip \labelsep {\bfseries #1}\hskip \labelsep {\bfseries #2.}]}{\end{trivlist}}
\newenvironment{exercise}[2][Exercise]{\begin{trivlist}
\item[\hskip \labelsep {\bfseries #1}\hskip \labelsep {\bfseries #2.}]}{\end{trivlist}}
\newenvironment{reflection}[2][Reflection]{\begin{trivlist}
\item[\hskip \labelsep {\bfseries #1}\hskip \labelsep {\bfseries #2.}]}{\end{trivlist}}
\newenvironment{proposition}[2][Proposition]{\begin{trivlist}
\item[\hskip \labelsep {\bfseries #1}\hskip \labelsep {\bfseries #2.}]}{\end{trivlist}}
\newenvironment{corollary}[2][Corollary]{\begin{trivlist}
\item[\hskip \labelsep {\bfseries #1}\hskip \labelsep {\bfseries #2.}]}{\end{trivlist}}
 
\begin{document}
 
% --------------------------------------------------------------
%                         Start here
% --------------------------------------------------------------
 
%\renewcommand{\qedsymbol}{\filledbox}
 
\title{HW2}%replace X with the appropriate number
\author{Carl Mueller\\ %replace with your name
CSCI 5254 - Convex Optimization} %if necessary, replace with your course title
 
\maketitle

\subsection*{3.1)}
\begin{proposition}{}
Suppose $f: R \rightarrow R$ is convex, and $a, b \epsilon dom f$ with $a < b$.
\end{proposition}
\subsubsection*{a)}
Show that:

$$
f(x) \le \frac{b-x}{b-a}f(a) + \frac{x-a}{b-a}f(b)
$$

\begin{proposition}{1}
Jensen's inequality: $f(\theta a + (1 - \theta)b) \le \theta f(a) + (1 - \theta) f(b)$
\end{proposition}
\begin{proof}
\begin{align*}
& \text{Let } x \in [a,b]\\
& b-x + x-a = b-a\\
& \frac{b-x}{b-a} + \frac{x-a}{b-a} = 1\\
& \text{Let } \theta = \frac{b-x}{b-a} \text{ then } 1 - \theta = \frac{x-a}{b-a}\\
& \text{Let } f(x) = f(\theta a + (1 - \theta)b)\\
& \text{Using Jenson's inequality:}\\
& f(\theta a + (1 - \theta)b) \le \theta f(a) + (1 - \theta) f(b)\\
& f(x) = f(\theta a + (1 - \theta)b) \le \theta f(a) + (1 - \theta) f(b) = \frac{b-x}{b-a}f(a) + \frac{x-a}{b-a}f(b)
\end{align*}
\end{proof}

\subsubsection*{b)}
Show that:
$\frac{f(x) - f(a)}{x-a} \le \frac{f(b) - f(a)}{b-a} \le \frac{f(b) - f(x)}{b-x}$

\begin{proof}
\begin{align*} 
& \text{Using inequality in a), we subtract $f(a)$ from both sides and simplify:}\\
& f(x) - f(a) \le \frac{b-x}{b-a}f(a) + \frac{x-a}{b-a}f(b) - f(a)\\
& \frac{f(x) - f(a)}{x-a} \le \frac{(1 - \frac{x-a}{b-a})}{x-a}-\frac{f(a)}{x-a} + \frac{x-a}{(x-a)(b-a)}f(b)\\
& \frac{f(x) - f(a)}{x-a} \le \frac{f(a)}{x-a} - \frac{f(a)}{b-a}-\frac{f(a)}{x-a} + \frac{f(b)}{b-a}\\
& \frac{f(x) - f(a)}{x-a} \le \frac{f(b)-f(a)}{b-a}\\
& \text{The same process can be done by subtracting $f(b)$ from both sides to get: }\\
& \frac{f(b)}{b-a} \le \frac{f(b)-f(x)}{b-x}\\
& \text{Combine inequalities: }\\
& \frac{f(x) - f(a)}{x-a} \le \frac{f(b) - f(a)}{b-a} \le \frac{f(b) - f(x)}{b-x}
\end{align*}
\end{proof}

\subsubsection*{c)}
Show that:
$ f'(a) \le \frac{f(b) - f(a)}{b-a} \le f'(b)$

\begin{proposition}{1}
$f'(a) = \lim_{x \to a} \frac{f(x)-f(a)}{x-a}$
$f'(b) = \lim_{x \to b} \frac{f(x)-f(b)}{x-b}$

\end{proposition}

\begin{proof}
\begin{align*} 
& \text{Let: }\\
& f(b) \ge f(a) + f'(a)(b-a)\\
& \frac{f(b) - f(a)}{b-a} \ge f'(a) \text{ where } b > a\\
& \text{and: }\\
& f(a) \ge f(b) + f'(b)(a-b)\\
& \frac{f(a) - f(b)}{a-b} \ge f'(b)\\
& \frac{f(b)-f(a)}{b-a} \le f'(b)\\
& f'(a) \le \frac{f(b) - f(a)}{b-a} \le f'(b)\\
\end{align*}
\end{proof}

\subsubsection*{d)}
Show that:
Suppose $f$ is twice differentiable. Use c) to show that $$\frac{f'(b)-f'(a)}{b-a} \ge 0$$


\begin{proof}
\begin{align*} 
& \text{Differentiate boths sides of result in part b): }\\
& f''(a) \le \frac{f'(b) - f'(a)}{b-a} \le f''(b)\\
& \text{Since } b-a \ge 0 \text{ it follows that :}\\
& 0 \le \lim_{b\to a}f'(a) = \lim_{b\to a}\frac{f'(b)-f'(a)}{b-a}
\end{align*}
\end{proof}

\subsection*{3.15)}

For $0 \le \alpha \le 1$ let $$ u_\alpha(x) = \frac{x^{\alpha} - 1}{\alpha}$$
We also define $u_0(x) = ln(x)$

\subsubsection*{a)}

Show that for $x > 0, u_0(x) = \lim_{\alpha\to 0}u_\alpha(x)$

\begin{proof}
\begin{align*} 
& \lim_{\alpha\to 0}u_\alpha(x)\\
& \text{L'Hopital's Rule: }\\
& \lim_{\alpha\to 0}x^{\alpha}ln(x)\\
& = ln(x) = u_0(x) \text{ by given definition.}
\end{align*}
\end{proof}

\subsubsection*{b)}

Show that $u_\alpha(x)$ is concave, monotone increasing, and all satisfy $u_\alpha(1) = 0$

\begin{proof}{Monotonicity and Concavity}
\begin{align*} 
& \text{Show that: } \nabla^2u_\alpha(x) \le 0 \\
& u'_\alpha(x) = x^{\alpha - 1} \ge 0 \text{ therefore monotonic increasing.} \\
& u''_\alpha(x) = (\alpha - 1)x^{\alpha - 2} \ge 0\\
& \alpha - 1 \text{ is always negative and } x^{\alpha - 2} \text{ is always postive, therefore:}\\
& u''_\alpha(x) \le 0 \text{ which implies concavity.}
\end{align*}
\end{proof}
\begin{proof}{$u_\alpha(1) = 0$}
\begin{align*} 
& \frac{1^\alpha - 1}{\alpha} = 0\ \forall \alpha, \alpha \in (0, 1)  
\end{align*}
\end{proof}

\subsection*{3.16)}
Show if concave, convex, quasiconcave, quasiconvex.
\subsubsection*{b)}
$$f(x) = x_1x_2$$
\begin{proof}{Convexity and Concavity}
\begin{align*}
& \nabla^2f(x_1,x_2) = 
\begin{bmatrix}
0 & 1 \\
1 & 0
\end{bmatrix},\
\nabla^2-f(x_1,x_2) = 
\begin{bmatrix}
0 & -1 \\
-1 & 0
\end{bmatrix} \\
& \text{Neither of which are positive semidefinite or negative semidefinite, respectively, therefore not convex nor concave}.
\end{align*}
\end{proof}
\begin{proposition}{}
Let $g$ be convex function $$g(x_1) = \frac{\alpha}{x_1}$$
\end{proposition}
\begin{proof}{Quasiconvecity \& Quasiconcavity}
\begin{align*}
& g'(x_1) = -\frac{\alpha}{x_1^2}\\
& g'(x_1) = \frac{3\alpha}{x_1^3} \ge 0\ \forall x_1 \in R_{++}\\
& epi(g(x_1)) = \set{(x_1, x_2) | frac{\alpha}{x_1} \le x_2}\\
& = \set{(x_1, x_2) | x_1x_2 \ge \alpha} \text{ which is the definition of the superlevel set of $f(x_1,x_2)$} \\
& \text{Since $g$ is convex, we know its epigraph is convex.}\\
& \text{Since $f$'s superlevel set is the epigraph of a convex function, we know it is convex}\\
& \text{Thus f is quasiconvex}
\end{align*}
\end{proof}

\subsubsection*{c)}
$$f(x) = 1/x_1x_2 \text{ on } R^2_{++}$$
\begin{proof}{Convexity and Concavity}
\begin{align*}
& \text{Calculate the Hessian:}\\
& \nabla^2f(x_1,x_2) = 
\begin{bmatrix}
\frac{2}{x_1^3x_2} & \frac{1}{x_1^2x_2^2} \\
\frac{1}{x_1^2x_2^2} & \frac{2}{x_1x_2^3}
\end{bmatrix} 
& \text{}
\end{align*}
\end{proof}


\subsubsection*{e)}
$$f(x) = x_1^2/x_2 \text{ on } RxR_{++}$$
\begin{proof}{Convexity and Concavity}
\begin{align*}
& \text{Calculate the Hessian:}\\
& \nabla^2f(x_1,x_2) = 
\begin{bmatrix}
\frac{2}{x_2} & -\frac{2x_1}{x_2^2} \\
-\frac{2x_1}{x_2^2} & \frac{2x_1^2}{x_2^3}
\end{bmatrix} \\
& \text{Using Sylvester's Criterion:}\\
& \frac{4x_1^2}{x_2^5} - \frac{4x_1^2}{x_2^r} > 0\\
& \text{Positve semidefinite matrix therefore convex!}\\
\end{align*}
\end{proof}

\subsection*{3.18)}
Use the proof of the log determinant to prove the following:

\subsubsection*{a)}

$f(X)= tr(X^{-1})$ is convex on $dom f = S^{n}_{++}$.

\begin{proof}
\begin{align*}
& \text{Using a linear affine funcion substitution:}\\
& g(t) = f(Z + tV) \text{ where } Z + tV \succ 0\\
& f(Z + tV) = tr((Z + tV)^{-1})\\
& = tr((Z^{1/2}(I + tZ^{1/2}V^{-1/2})Z^{-1/2})^{-1})\\
& = tr((Z^{1/2})^{-1}(I + tZ^{1/2}V^{-1/2})Z^{-1/2})^{-1})\\
& \text{Using Spectral Decomposition: $\hat{V} = Z^{1/2}VZ^{-1/2} = Q\Lambda Q^T$}\\
& = tr((Z)^{-1}(I + tQ\Lambda Q^T)^{-1})\\
& = tr(Z^{-1}(Q^T)^{-1}(I + t\Lambda)^{-1}Q^{-1})\\
& = tr(Z^{-1}Q(I + t\Lambda)^{-1}Q^T)\\
& = tr(Q^TZ^{-1}Q(I + t\Lambda)^{-1})\\
& = \sum_{i,i}(Q^TZ^{-1})(\frac{1}{1+t\lambda_i})\\
& \text{$(Q^TZ^{-1})_{ii}$ is always postive}\\
& \text{$(\frac{1}{1+t\lambda_i})$ is a sum of convex functions}\\
& \text{Therefore $f(X)$ is convex.}
\end{align*}
\end{proof}

\subsubsection*{b)}

$f(X)= (det(X))^{1/n}$ is concave on $dom f = S^{n}_{++}$.

\begin{proof}
\begin{align*}
& \text{Using a linear affine funcion substitution:}\\
& g(t) = f(Z + tV) \text{ where } Z + tV \succ 0\\
& f(g(t)) = (det(Z + tV))^{1/n}\\
& f(g(t)) = (det(^{1/2}(I + tZ^{1/2}V^{-1/2})Z^{-1/2}))^{1/n}\\
& f(g(t)) = (det(Z(I + tQ\Lambda Q^T))^{1/n}\\
& f(g(t)) = (det(Z)det(I + tQ\Lambda Q^T))^{1/n}\\
& f(g(t)) = det(Z)^{1/n}det(Q^TQ(I + t\Lambda))^{1/n}\\
& f(g(t)) = det(Z)^{1/n}\prod_i(I + t\lambda_i)^{1/n}\\
& \text{$\prod_i(I + t\lambda_i)^{1/n}$ is the gemotric mean which is concae thus $f$ is concave.}
\end{align*}
\end{proof}

\subsection*{3.19)}
\subsubsection*{a)}

Show $f(x) = \sum_{}^{}\alpha_ix_i$ is convex where $\alpha_1 \ge \alpha_2 \ge \dots \ge 0$.
and $x_i$ demnnotes the i'th largest component. Hint: $f(x) = \sum_{i=1}^{k}x_i$ is convex on $R^n$

\begin{proof}
\begin{align*}
& f(x) = \alpha_r(x_1 + x_2 + \dots + x_r) + (\alpha_{r-1} - \alpha_r)(x_1 + x_2 + \dots + x_{r-1}) + \dots\\
& \text{$(x_1 + x_2 + \dots + x_r)$ is convex based of the hint.} \\
& \text{$\alpha_{r-1} - \alpha_r$ is positive.}\\
& \text{Therefore $f$ is a sum on nonnegative convex functions}\\
\end{align*}
\end{proof}

\subsection*{3.22)}
Show the following functions are convex.

\subsubsection*{b)}
$f(x,u,v) = - \sqrt{uv - x^Tx}$ on $dom f = \set{(x,u,v) | uv > x^Tx, u,v >0}$

\begin{proposition}{}
$\frac{x^Tx}{u}$ and $-\sqrt{x_1x_2}$ is convex on $R_{++}^{2}$.
\end{proposition}

\begin{proof}
\begin{align*}
& \text{Let } h(x_1,x_2) = -\sqrt{x_1x_2}\\
& \text{Let } g_1(x, u, v) = u \rightarrow concave.\\
& \text{Let } g_2(x, u, v) = v - \frac{x^Tx}{u} \rightarrow concave.\\
& \text{$h(g_1,g_2) = f$ is a concave if h is concave $\hat{h}$ is nonincreasing and $g_1,g_2$ are both concave.}\\
& \text{Thus f is convex.}\\
\end{align*}
\end{proof}

\subsubsection*{c)}
$f(x,u,v) = -log(uv - x^Tx)$ on $dom f = \set{(x,u,v) | uv > x^Tx, u,v >0}$

\begin{proposition}{}
$\frac{x^Tx}{u}$ and $-\sqrt{x_1x_2}$ is convex on $R_{++}^{2}$.
\end{proposition}
\begin{proof}
\begin{align*}
& \text{Let } h(x_1,x_2) = -log(x_1x_2) \rightarrow \text{convex and nonincreasing.}\\
& \text{Let } g_1(x, u, v) = u \rightarrow \text{concave.}\\
& \text{Let } g_2(x, u, v) = v - \frac{x^Tx}{u} \rightarrow \text{concave.}\\
& \text{$h(g_1,g_2) = f$ is a convex if h is concave $\hat{h}$ is nonincreasing and $g_1,g_2$ are both concave.}\\
& \text{Thus f is concave.}
\end{align*}
\end{proof}

\subsection*{3.24)}
Determine if the following are xcconvex, concave, quasiconvex, or quasiconcave.

\subsubsection*{c)}
$\set{a_1, \dots , a_n}$ where $a_1 < a_2 < \dots < a_n$ with $prob(x=a_i)=p_i$.

$$prob(\alpha \le x \le \beta)$$
\begin{proof}
\begin{align*}
& prob(\alpha \le x \le \beta) = \sum_{i=j}^{k}p_i\\
& \text{This is a sum of linear function in the probability space, each of which are convex and nonegative.}
& \text{As a sum of nonegative linear functions, the function is linear.}
\end{align*}
\end{proof}


\subsubsection*{h)}
$\set{\beta - \alpha | prob(\alpha \le x \le \beta) \ge 0.9}$

\begin{proposition}{}
Let $f(p)$ be a function that returns an interval for the given propbability.
Then $f(\sum_{k=i}^{j}) = a_j - a_i$ and $f(.9) = \gamma$
\end{proposition}
\begin{proof}
\begin{align*}
& \set{\beta - \alpha | prob(\alpha \le x \le \beta) \ge 0.9}\\
& \set{\beta - \alpha | \sum_{i=j}^{k}p_i \ge 0.9}\\
& \text{We need to find a $f(p) \ge \gamma$}\\
&\text{Thus quasiconcanv iff all invervals of width less that $\gamma$ have $p < .9$}\\
& \sum_{k=i}^{j}p_k < .9
\end{align*}
\end{proof}




\subsection*{3.26)}
\subsubsection*{a}

Show $\sum_{i=1}^{k}\lamba_i(x)$ is convex on $S^n$ where 
$$\sum_{i=1}^{k}\lamba_i(x) = sup \set{tr(v^Txv)| v \in R^{n \times k}, v^Tv = I}$$

\begin{proof}
\begin{align*}
& tr(v^Txv) \text{ is a linear function}\\
& sup{tr(v^Txv)} \text{is a pointwise supremum of linear functions which is convex thus the function is convex.}
\end{align*}
\end{proof}


\subsection*{3.35)}
$S_C(y) = sup\set{y^Tx |x \in C}$

\subsubsection*{a}
Show that $S_B = S_{convB}$.

I.e. that:

$$sup\set{y^Tu | u \in C} = sup\set{y^Tv | v \in conv(C)}$$

\begin{proof}
Disprove the negative:
\begin{align*}
& sup\set{y^Tu | u \in C} < sup\set{y^Tv | v \in conv(C)}\\
& y^Tu < y^Tv\\
& y^Tu < y^T\sum_{i}\theta_i u_i\\
& \text{Thus can't be true for all i since there must be one point where $u=v$.}\\
& \text{Thus we must conclude equality since the same contradition holds for $>$.} 
\end{align*}
\end{proof}

\subsubsection*{b}
Let $B$ be close and convex. Show that $A \subseteq B \Leftrightarrow S_A(y) \le S_B(y)$.

\begin{proof}{$A \subseteq B \Rightarrow sup\set{y^Tu | u \in A} \le sup\set{y^Tv | v \in B}$}
\begin{align*}
& \text{Show less than:}\\
& y^Tu < y^Tv;\ \text{ bebcause there exists$\exists v \in B$ where $v < u$ since $A \subseteq B$ for some $y$}\\
&\text{Show equality:}\\
& y^Tu = y^Tv \text{ because $\exists v \in B$ where $v = u$ since $A \subseteq B$}\\
&\text{Show greater than is not true.}\\
& y^tu > y^Tv \text{ but because $\exists v \in B$ where $v > u$ since $A \subseteq B$}\\
& \text{this contradicts the strict $>$ inequality.}\\
\end{align*}
\end{proof}

\begin{proof}{$A \not\subseteq B \Rightarrow sup\set{y^Tu | u \in A} \ge sup\set{y^Tv | v \in B}$}
\begin{align*}
&  \text{If $x \in A$ and $x \not\in B$, then there exists a support hyperplane that separates B from x.}\\
& \text{This means that $y^Tx > y^Ty\ \forall y \in B$.}\\
\end{align*}
\end{proof}

\subsection*{3.36)}
Derive the conjugates for the following functions.

\subsubsection*{a)}
$f(x) = max_{i=1}x_i$ on $R^n$

$$f^*(y) = f^*(y) = sup(y^Tx - max_{i=1}x_i)$$
\begin{proof}{$y \prec 0$}
\begin{align*}
& ty_k \rightarrow \infty \text{ if } t \rightarrow \infty\\
& \text{Thus } sup(y^Tx - max_{i=1}x_i) \text{ DNE so $y \preceq 0$ not in the domain.}
\end{align*}
\end{proof}

\begin{proof}{$y \succeq 0$}
\begin{align*}
& f^*(y) = sup(y^Tx - max_{i=1}x_i)\\
& \text{Say } x = t\overrightarrow{1}\\
& \text{Substitute into $f$:}\\
& t\overrightarrow{1}^Ty-max_{i=1}t\overrightarrow{1}\\
& t\overrightarrow{1}-t\\
& \text{Where } \overrightarrow{1}^Ty \text{ must be = 1 to be bounded since $t\in(-\infty, \infty)$}\\
& \text{Thus $y \succeq 0$ in the domain}
\end{align*}
\end{proof}

\subsubsection*{d)}
$f(x) = x^p$ on $R_{++}$, where $p>1$ or $p<0$.
$$f^*(y) = f^*(y) = sup(xy - x^p)$$
\begin{proof}{$y < 0$}
\begin{align*}
& y = x^{p-1} \text{ where } x\not=1 \text{ is bounded.}
\end{align*}
\end{proof}

\begin{proof}{$y > 0$}
\begin{align*}
& \text{Find the point where the rate of change of the difference is 0 and thus bounded:}\\
& \frac{d}{dx}yx = \frac{d}{dx}x^p\\
& y = px^{p-1}\\
& (\frac{y}{p})^{\frac{1}{p-1}} = x\\
& \text{The function value for that value of x is: }\\
& f^*(y) = (p-1)(\frac{y}{p})^{\frac{p}{p-1}}
\end{align*}
\end{proof}

% --------------------------------------------------------------
%     You don't have to mess with anything below this line.
% --------------------------------------------------------------
 
\end{document}