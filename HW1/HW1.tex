
% This is all preamble stuff that you don't have to worry about.
% Head down to where it says "Start here"
% --------------------------------------------------------------
 
\documentclass[12pt]{article}
 
\usepackage[margin=1in]{geometry} 
\usepackage{amsmath,amsthm,amssymb}
\usepackage{braket}
 
\newcommand{\N}{\mathbb{N}}
\newcommand{\Z}{\mathbb{Z}}
 
\newenvironment{theorem}[2][Theorem]{\begin{trivlist}
\item[\hskip \labelsep {\bfseries #1}\hskip \labelsep {\bfseries #2.}]}{\end{trivlist}}
\newenvironment{lemma}[2][Lemma]{\begin{trivlist}
\item[\hskip \labelsep {\bfseries #1}\hskip \labelsep {\bfseries #2.}]}{\end{trivlist}}
\newenvironment{exercise}[2][Exercise]{\begin{trivlist}
\item[\hskip \labelsep {\bfseries #1}\hskip \labelsep {\bfseries #2.}]}{\end{trivlist}}
\newenvironment{reflection}[2][Reflection]{\begin{trivlist}
\item[\hskip \labelsep {\bfseries #1}\hskip \labelsep {\bfseries #2.}]}{\end{trivlist}}
\newenvironment{proposition}[2][Proposition]{\begin{trivlist}
\item[\hskip \labelsep {\bfseries #1}\hskip \labelsep {\bfseries #2.}]}{\end{trivlist}}
\newenvironment{corollary}[2][Corollary]{\begin{trivlist}
\item[\hskip \labelsep {\bfseries #1}\hskip \labelsep {\bfseries #2.}]}{\end{trivlist}}
 
\begin{document}
 
% --------------------------------------------------------------
%                         Start here
% --------------------------------------------------------------
 
%\renewcommand{\qedsymbol}{\filledbox}
 
\title{HW1}%replace X with the appropriate number
\author{Carl Mueller\\ %replace with your name
CSCI 5254 - Convex Optimization} %if necessary, replace with your course title
 
\maketitle

\subsection*{2.11 a)}
Show that the hyperbolic set $\{x\ \in\ R^2_+\ |\ \prod^{n}_{i=1}\ x_i\ \ge\ 1\}$ is convex for $n\ = \ 2$.
 
\begin{proposition}{1}
Consider the convex combination of two points $x,\ y\ \in\ R^2_+$ called $z$.\\
\begin{align*}
z=\theta x + (1-\theta) y\\
\end{align*}
where the components of $z$ are:\\
\begin{align*}
z_1=\theta x_1 + (1-\theta) y_1\\
z_2=\theta x_2 + (1-\theta) y_2\\
\end{align*}
\end{proposition}

\begin{proposition}{2}
If $x \succeq y$ the $z \succeq y$ and $ z_1z_2 \ge y_1y_2 \ge 1$
\end{proposition}

\begin{proposition}{3}
If $y \succeq x$ the $z \succeq x$ and $ z_1z_2 \ge x_1x_2 \ge 1$
\end{proposition}

\begin{proposition}{4}
If $x \not\succeq y$ then either $(y_1 - x_1)$ or $(y_2 - x_2)$ is negative.
Therefore $(y_1 - x_1)(y_2 - x_2) < 0$.
\end{proposition}

\begin{proof}
We show that $z_1z_2\ \ge\ 1$:

\begin{align*}
& (\theta x_1 + (1-\theta) y_1)(\theta x_2 + (1-\theta) y_2) \ge 1\\
& = \theta^2x_1x_2 + (1-\theta)^2y_1y_2 + \theta(1-\theta)x_1y_2 + \theta(1-\theta)x_2y_1\\
\text{After some ugly algebra:}\\
& = \theta x_1x_2 + (1-\theta)y_1y_2 - \theta(1-\theta)(y_1-x_1)(y_2-x_2)\\
\end{align*}
\begin{align*}
\theta x_1x_2 + (1-\theta)y_1y_2 \ge 1
\end{align*}
Based on proposition 4, the following is also true:
\begin{align*}
- \theta(1-\theta)(y_1-x_1)(y_2-x_2) \ge 0
\end{align*}
Therefore:
\begin{align*}
\theta x_1x_2 + (1-\theta)y_1y_2 - \theta(1-\theta)(y_1-x_1)(y_2-x_2) \ge 1
\end{align*}
Thus proving convexity.
\end{proof}
 
\subsection*{2.11 b)}
Show that the hyperbolic set $\{x\ \in\ R^2_+\ |\ \prod^{n}_{i=1}\ x_i\ \ge\ 1\}$ is convex in the general case.\\

\begin{proposition}{1}
If $a\ ,b\ \ge\ 0$ and $0\ \le\ \theta\ \le\ 1$ then $a^{\theta}b^{1 - \theta}\ \le\ \theta a\ +\ (1\ -\ \theta b)$
\end{proposition}

\begin{proposition}{2}
$\prod_{i}x_i\ \ge\ 1$
\end{proposition}

\begin{proposition}{3}
$\prod_{i}y_i\ \ge\ 1$
\end{proposition}

\begin{proof}
Given $z$ as a convex combination: $z=\theta\ x\ +\ (1-\theta)\ y$, show that $$\prod_{i}\ \theta\ x_i\ +\ (1-\theta)\ y_i\ \ge\ 1$$\\\\

Given proposition 1:
\begin{align*}
& \prod_{i} x_i^\theta y_i^\theta  \le \prod_{i} \theta x_i + (1-\theta) y_i = \prod_{i}z_i\\
& \prod_{i} x_i^\theta y_i^{(1-\theta)}\\
& = (\prod_{i}x_i)^\theta (\prod_{i}y_i)^{(1-\theta)}
\end{align*}
Given propositions 2 and 3:
\begin{align*}
&(\prod_{i}x_i)^\theta (\prod_{i}y_i)^{(1-\theta)} \ge 1
\end{align*}
Since either term $\prod_{i}x_i$ and $\prod_{i}y_i$ is $\ge\ 1$.
Thus:
\begin{align*}
& 1 \le \prod_{i} x_i^\theta y_i^\theta  \le \prod_{i} \theta x_i + (1-\theta) y_i = \prod_{i}z_i\\
\end{align*}
Which therefore proves convexity in the general.
\end{proof}

\subsection*{2.12 c)}
The set $\{x\ R^n\ |\ a_1^Tx\ \le\ b_1,\ a_2^Tb_2\ \le\ b_2\}$ is the intersection of two halfspaces:
\begin{align*}
a_1^Tx \le b_1\\
a_2^Tx \le b_2
\end{align*}
Halfspaces are convex therefore a wedge is a intersection of convex sets and therefore is convex.

\subsection*{2.12 e)}
The set $$\{x\ |\ dist(x,S)\ \le\ dist(x,T)\}$$ is not generally convex.
For example, consider the set $S\ =\ \{x\ | x=(0,0)\}$ (the origin) and the complement set of a ball $T\ =\ \{x\ | U\backslash B(x,r), x=[0,0]\}$ centered at the origin. The original set creates a washer between the ball's complement which is not convex.

\subsection*{2.12 f)}
The set $\{x\ |\ x\ +\ S_2\ \subseteq\ S_1 \}$ where $S_1$ is convex. If $y\ \in\ S_2$ then $x\ +\ y\ \in\ S_1$. Then the original set: $$\{x\ |\ x\ +\ S_2\ \subseteq\ S_1 \}\ = \cap\ \{x\ |\ x\ +\ y\ \subseteq\ S_1\} \forall y \in S_2$$
Thus the intersection of all sets of $S_1$ shifted by $-y$ each of which are convex.

\subsection*{2.12 g)}
The set: $\set{x |\ ||x-a||_2 \le \theta||x-b||_2}$:
\begin{align*}
\{x\ |\ ||x-a||_2\ \le\ \theta||x-b||_2\}\\
= \{x\ |\ ||x-a||_2^2\ \le\ \theta^2||x-b||_2^2\}\\
\text{Using definition of a ball:}\\
= \{x\ |\ (x-a)T(x-a)\ \le\ \theta^2(x-b)^T(x-b)\}\\
= \{x\ |\ (1-\theta^2)x^Tx - 2(a-\theta^2b)^Tx\ +\ a^Ta\ -\ \theta^2b^Tb \le\ 0\}\\
\end{align*}
If $\theta = 1$ then the remaining inequality indicates a halfspace.
For all other values, we're left with a ball. All cases of which are convex.

\subsection*{2.14 a)}
Given $S \subseteq R^n$, let $||.||$ be a norm in $R^n$, $a \ge 0$, and the set:

\begin{align*}
S_a = \{x\ |\ dist(x,S) \le\ a\}\ \text{where}\ dist(x,S) = \textbf{inf}_{y\in S}||x-y|| 
\end{align*}
Show if $S$ is convex then $S_a$ is convex.\\


\begin{proof}
\begin{align*}
S_a = \set{x | dist(x,S) \le a} \text{where}\ dist(x,S) = \textbf{inf}_{y\in S}||x-y||\\
\text{Substitute line segment between $x_1$ \& $x_2$ into the constraint and show $\le\ a$:}\\
\textbf{inf}_{y\in S}|| \theta x_1 + (1-\theta)x_2-y_o||\\
\text{Create convex combination representing $y_o$ and substitute:}\\
\textbf{inf}_{y\in S}|| \theta x_1 + (1-\theta)x_2-\theta y_1 + (1-\theta)y_2||\\
= \textbf{inf}_{y\in S}|| \theta (x_1 - y_1) + (1-\theta)(x_2 - y_2)||\\
\text{Triangle inequality and scalability of norm:}\\
\le (\theta) \textbf{inf}_{y\in S}||(x_1 - y_1)|| + (1-\theta)\textbf{inf}_{y \in S}||x_2 - y_2||\\
\le a
\end{align*}
Thus the set is convex.
\end{proof}

\subsection*{2.14 b)}
Given $S \subseteq R^n$, let $||.||$ be a norm in $R^n$, $a \ge 0$, and the set:

\begin{align*}
S_{-a} = \set{x | B(x,a) \subseteq S}
\end{align*}
where $B(x,a)$ is the ball in the norm $||.||$ with center $x$ and radius $a$. 
Show if $S$ is convex. $S_{-a}$ is convex.
\begin{align*}
S_{-a}  = \set{x | ||y-x|| \le a}\\
\text{Let $x_o = \theta x_1 + (1-\theta)x_2\ \in S_{-a}$}\\
\text{Let $y_o = \theta y_1 + (1-\theta)y_2\ \in S$}\\
S_{-a}  = \set{x | ||\theta y_1 + (1-\theta)y_2 - \theta x_1 + (1-\theta)x_2|| \le a}\\
\text{Triangle inequality and scalability of norm:}\\
= \theta||(y_1 - x_1)|| + (1-\theta)||(y_2-x_2)||
\le a
\end{align*}
Thus the set is convex.

\subsection*{2.15 a)}
Given $\alpha \le E[f(x)] \le \beta$ where $\textbf{prob}(x_i = a_i) = p_i$ and $E[f(x)]= \sum_{i=1}^{n}p_if(a_i)$.
Also the set $P = \set{p | \textbf{1}^TP = 1. p_i > 0}$ is the intersection of a set of halfsapces ($p_i>0$) and a hyperplane($\textbf{1}^TP = 1$) creating a polyhedron and thefore is convex.
\begin{align*}
\alpha \le E[f(x)] \le \beta\\
\alpha \le E[f(x)]\ \cap\ E[f(x)] \le \beta\\
\alpha \le \sum_{i=1}^{n}p_if(a_i)\ \cap\ \sum_{i=1}^{n}p_if(a_i) \le \beta\\
\end{align*}
Each of the above inequalities are inequalities in P since the $E[]$ function lies within the space P.
Therfore both these inqualities are halfspaces in P, and are convex\\
Thus $$\alpha \le \sum_{i=1}^{n}p_if(a_i)\ \cap\ \sum_{i=1}^{n}p_if(a_i) \le \beta\\$$ is an intersection of halspaces and is therefore convex.

\subsection*{2.15 b)}
Given $\alpha \le E[f(x)] \le \beta$ where $\textbf{prob}(x_i \ge \alpha)$.
Again the set $P = \set{p | \textbf{1}^TP = 1, p_i > 0}$ is the intersection of a set of halfsapces ($p_i>0$) and a hyperplane ($\textbf{1}^TP = 1$) creating a polyhedron and thefore is convex.
\begin{align*}
\textbf{prob}(x_i \ge \alpha) = \sum_{x_i\ge\alpha}p_i \le \beta
\end{align*}
This is another inquality in the space of P. Since P is convex, therefore the halspace in P is convex and thus the original probability set is convex.

\subsection*{2.15 f)}
Given $\textbf{var}(x)\le \alpha$.

Let $a = [.75, .25]$, $p = [.5, .5]$, $\alpha = .05$.
\begin{align*}
\textbf{var}(x)\le \alpha\\
\sum_{i=1}^{n}p_ia_i^2 - (\sum_{i=1}^{n}p_ia_i)^2 \le \alpha\\
\text{Plug in values $a$, $p$ \& $\alpha$:}\\
(.5*.75^2+.5*.25^2)- (.5*.75+.5*.25)^2\\
= (.3125) - (.25)\\
=.0625 \not\le .05
\end{align*}
Thus not convex.

\subsection*{2.15 g)}
Given $\textbf{var}(x)\ge \alpha$.

\begin{align*}
\textbf{var}(x)\le \alpha\\
\sum_{i=1}^{n}p_ia_i^2 - (\sum_{i=1}^{n}p_ia_i)^2 \ge \alpha\\
p^T(a^2) - (p^Ta)^2 \ge \alpha\\
p^T(a^2) - (p^Taa^Tp) \ge \alpha\\
\end{align*}
Since $aa^T$ makes a positive semi-definite matrix, the above defines a convex set.

\subsection*{2.19 a)}
Let 
$$f: R^m -> R^n$$
$$f(x) = \frac{Ax+b}{c^Tx +d}$$
$$\textbf{dom}f = \set{x | c^Tx + d > 0}$$
$$C = \set{y | g^Ty \le h},\ \text{where C is convex.}$$
$$f^{-1}(C) = \set{x \in \textbf{dom}f\ |\ f(x) \in C}$$ 
Describe $f^{-1}(C)$:
\begin{proof}
\begin{align*}
f^{-1}(C) = \set{x\ |\ g^Tf(x) \in C}\\
\text{Apply the affine function:}\\
= \set{x |\ g^T\frac{Ax+b}{c^Tx +d} \le h, c^Tx + d > 0}\\
= \set{x |\ g^TAx+g^Tb \le hc^Tx + hd, c^Tx + d > 0}\\
\end{align*}
This is the intersection of two half spaces:
$$ \set{x | c^Tx + d > 0} \cap \set{x |\ g^TAx+g^Tb \le hc^Tx + hd, c^Tx + d > 0}$$
\end{proof}

\subsection*{2.19 b)}
Let 
$$f: R^m -> R^n$$
$$f(x) = \frac{Ax+b}{c^Tx +d}$$
$$\textbf{dom}f = \set{x | c^Tx + d > 0}$$
$$C = \set{y | Gy \preceq h},\ \text{where C is convex.}$$
$$f^{-1}(C) = \set{x \in \textbf{dom}f\ |\ f(x) \in C}$$ 
Describe $f^{-1}(C)$:
\begin{proof}
\begin{align*}
f^{-1}(C) = \set{y | Gy \preceq h}\\
\text{Apply the affine function:}\\
= \set{x \in \textbf{dom}f | Gf(x) \preceq h}\\
= \set{x | G\frac{Ax+b}{c^Tx +d} \le h, c^Tx + d > 0}\\
= \set{x | GAx+Gb \le hc^Tx + hd, c^Tx + d > 0}\\
\end{align*}
We subtract $Gb$ and $hc^Tx$ from both sides to have the "matrix $\preceq$ vector" form.
\begin{align*}
f^{-1}(C) = \set{x | (GA-hc^T)x \preceq hd-Gb, c^Tx+d > 0}
\end{align*}
This is again a polyhedron intersecting the domain of f which is a half space.
\end{proof}

\subsection*{2.33 a)}
Given $$K_{m+} = \set{x \in \textbf{R}^n | x_1 \ge x_2 \ge ... \ge x_n \ge 0}$$
Show $K_{m+}$ is proper:\\

Convexity:
$K_{m+}$ is a system of inequalities $x_1 \ge x_2 \ge ... \ge x_n \ge 0$ which  constitutes a polyhedron.

Closed:
Since the system of inequalities are not strict, the polyhedron is closed.

Interior:
The point $x = [5, 4, 3, 2, 1]$ meets the requirements of the constraints with strict inequality and thefore represents an interior point.

Pointed:
A cone is pointed if $x \in K_{m+}$ \& $-x \in K_{m+}$ if $x = 0$.
Say $x = [1, 0]$ and $-x = [-1, 0]$. $-x$ violates inequality constraints. Only if $x, -x=0$ are the constraints not violated.


\subsection*{2.33 b)}
Find the dual cone $K_{m+}^*$

Given the definition of the dual cone:$$K^* = \set{y|x^Ty \ge 0\ \forall x \in K}$$ Show that $$\sum_{i=1}^{n}x_iy_i \ge 0\ \forall x \in K_{m+}$$:

\begin{align*}
\sum_{i=0}^{n}x_iy_i = (x_1 - x_2)y_1 + (x_2-x_3)(y_1 + y_2) + \dots + (x_{n_1}-x_n)(y_i + \dots + y_{n+1})\\
\text{we know $(x_{n_1}-x_n) \ge 0$ based on the constraints on $K_{m+}$}\\
\therefore x_Ty \ge 0 \iff y_1,\ y_1+y_2,\ y_1 + \dots + y_n \ge 0\\
\therefore K^*_{m+} = \set{y | \sum_{i=1}^{n} y_i \ge 0, k=1, \dots, n}
\end{align*}

\subsection*{2.3)}
If we can recursively show that the midpoint of the midpoint and the end of a line segment is contained by the set for all line segments, then midpoint convexity implies convexity.
For example:
$$\dfrac{x_1+\dfrac{x_1+x_2}{2}}{2} = \dfrac{x_1}{2}+\dfrac{x_1}{4}+\dfrac{x_2}{4} = 3/4x_1 + 1/4x_2\in C$$
And the next recursion:
$$\dfrac{x_1+\dfrac{x_1+\dfrac{x_1+x_2}{2}}{2}}{2} = \dfrac{x_1}{2}+\dfrac{x_1}{4}+\dfrac{x_1}{8} +\dfrac{x_2}{8} = 7/8x_1 +1/8x_2\in C$$

The coefficients always add to one, meeting the requirement of convexity for all line segments and the recursively reassigned midpoints. The general pattern shows a consistent convex combination of which the points is always contained in C:

$$(\dfrac{2^k-1}{2^k})x_1 +(1 - \dfrac{2^k-1}{2^k})x_2\in C$$

As $k \rightarrow \infty$:

$$lim_{k \rightarrow \infty} (\dfrac{2^k-1}{2^k})x_1 +(1 - \dfrac{2^k-1}{2^k})x_2 = x_1 \in C$$


% --------------------------------------------------------------
%     You don't have to mess with anything below this line.
% --------------------------------------------------------------
 
\end{document}