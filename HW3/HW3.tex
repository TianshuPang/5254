% --------------------------------------------------------------
% This is all preamble stuff that you don't have to worry about.
% Head down to where it says "Start here"
% --------------------------------------------------------------
 
\documentclass[12pt]{article}
 
\usepackage[margin=1in]{geometry} 
\usepackage{amsmath,amsthm,amssymb}
\usepackage{braket}
 
\newcommand{\N}{\mathbb{N}}
\newcommand{\Z}{\mathbb{Z}}
 
\newenvironment{theorem}[2][Theorem]{\begin{trivlist}
\item[\hskip \labelsep {\bfseries #1}\hskip \labelsep {\bfseries #2.}]}{\end{trivlist}}
\newenvironment{lemma}[2][Lemma]{\begin{trivlist}
\item[\hskip \labelsep {\bfseries #1}\hskip \labelsep {\bfseries #2.}]}{\end{trivlist}}
\newenvironment{exercise}[2][Exercise]{\begin{trivlist}
\item[\hskip \labelsep {\bfseries #1}\hskip \labelsep {\bfseries #2.}]}{\end{trivlist}}
\newenvironment{reflection}[2][Reflection]{\begin{trivlist}
\item[\hskip \labelsep {\bfseries #1}\hskip \labelsep {\bfseries #2.}]}{\end{trivlist}}
\newenvironment{proposition}[2][Proposition]{\begin{trivlist}
\item[\hskip \labelsep {\bfseries #1}\hskip \labelsep {\bfseries #2.}]}{\end{trivlist}}
\newenvironment{corollary}[2][Corollary]{\begin{trivlist}
\item[\hskip \labelsep {\bfseries #1}\hskip \labelsep {\bfseries #2.}]}{\end{trivlist}}
 
\begin{document}
 
% --------------------------------------------------------------
%                         Start here
% --------------------------------------------------------------
 
%\renewcommand{\qedsymbol}{\filledbox}
 
\title{HW3}
\author{Carl Mueller\\ %replace with your name
CSCI 5254 - Convex Optimization} %if necessary, replace with your course title
\maketitle

\subsection*{4.1)}

\begin{equation*}
\begin{aligned}
& \underset{x}{\text{minimize}}
& & f_0(x_1, x_2)\\
& \text{subject to}
& & 2x_1 + x_2 \ge 1\\
&&& x_1 + 3x_2 \ge 1\\
&&& x_1 \ge 0, x_2 \ge 0
\end{aligned}
\end{equation*}
Make a sketch of the feasible set:\\\\\\\\\\\\

\subsubsection*{a)}

\begin{equation*}
\begin{aligned}
& f_0(x_1, x_2)= x_1 + x_2\\
& p^* = inf\set{x_1 + x_2 | 2x_1 + x_2 \ge 1, x_1 + 3x_2 \ge 1, x_1 \ge 0, x_2 \ge 0}\\
& p^* = 3/5\\
& x^* \in \set{(2/5, 1/5)}
\end{aligned}
\end{equation*}

\subsubsection*{b)}

\begin{equation*}
\begin{aligned}
& f_0(x_1, x_2)= -x_1 - x_2\\
& p^* = inf\set{-x_1 - x_1 | 2x_1 + x_2 \ge 1, x_1 + 3x_2 \ge 1, x_1 \ge 0, x_2 \ge 0}\\
\end{aligned}
\end{equation*}
No lowerbound as $x \rightarrow \infty$ \& $x_2 \rightarrow \infty$ is in the feasible set then $-x_1 - x_2 \rightarrow -\infty$

\subsubsection*{c)}

\begin{equation*}
\begin{aligned}
& f_0(x_1, x_2)= x_1\\
& p^* = inf\set{x_1 | 2x_1 + x_2 \ge 1, x_1 + 3x_2 \ge 1, x_1 \ge 0, x_2 \ge 0}\\
& p^* = 0\\
& x^* \in \set{(0, x_2)}
\end{aligned}
\end{equation*}

\subsubsection*{d)}

\begin{equation*}
\begin{aligned}
& f_0(x_1, x_2)= max(x_1, x_2)\\
& p^* = inf\set{max(x_1, x_2) | 2x_1 + x_2 \ge 1, x_1 + 3x_2 \ge 1, x_1 \ge 0, x_2 \ge 0}\\
& \text{Say } x_1 = x_2\\
& 2x_1 = 1- x_1\\
& x_1 = 1/3\\
& \therefore\\
& p^* = 1/3\\
& x^* \in \set{(1/3, 1/3)}
\end{aligned}
\end{equation*}

\subsubsection*{e)}

\begin{equation*}
\begin{aligned}
& f_0(x_1, x_2)= x_1^2 + 9x_2^2\\
& p^* = inf\set{max(x_1, x_2) | 2x_1 + x_2 \ge 1, x_1 + 3x_2 \ge 1, x_1 \ge 0, x_2 \ge 0}\\
& \text{Let } 2x_1 + x_2 = 1, x_1 + 3x_2 = 1\\
& 2x_1 + x_2 = 1\\
& x = (1/3, 1/3)\\
& x_1 + 3x_2 = 1\\
& x = (1/2, 1/6) \\
& \text{ This gives the smallest $p^*$ and satisfies all constraints} \\
& \therefore\\
& p^* = 1/2\\
& x^* \in \set{(1/2, 1/6)}
\end{aligned}
\end{equation*}

\subsection*{4.3)}

\text{We use the optimality criterion: }
$$\nabla_{f_0}(x^*)^T(y-x) \ge 0,\ \forall y \in x, x \in \text{feasible set}$$
\begin{equation*}
\begin{aligned}
& \nabla{f_0}(x^*)\\
& = ([1, 1/2, -1]\cdot \begin{bmatrix}
    13 & 12 & -2 \\
    12 & 17 & 6 \\
    -2 & 6 & 12 
    \end{bmatrix} + \begin{bmatrix}
    -22\\
    -14.5\\
    13.0 
    \end{bmatrix})\cdot \begin{bmatrix}
    y_1 - 1\\
    y_2 - 1/2\\
    y_3 + 1 
    \end{bmatrix}\\
& = [-1, 0, 2]\cdot \begin{bmatrix}
    y_1 - 1\\
    y_2 - 1/2\\
    y_3 + 1 
    \end{bmatrix}\\
& = -1(y_1 - 1)+ 2(y_2 + 1) \ge 0
\end{aligned}
\end{equation*}
This statistfies the optimality condition.

\subsection*{4.7)}

\subsubsection*{a)}

\begin{equation*}
\begin{aligned}
& f_0(x) \text{ is convex}\\
& \text{Show } frac{f_0(x)}{c^Tx +d} \text{ is quasiconvex.}\\
& \set{x | frac{f_0(x)}{c^Tx +d} \le \alpha}\\
& \set{x | f_0(x) \le \alpha(c^Tx +d)}\\
& \set{x | f_0(x) \le \hat{\alpha}}\\
& \text{Since $f_0(x)$ is convex, all its level sets are convex}\\
& \therefore\\
& \frac{f_0(x)}{c^Tx +d} \text{ is quasiconvex.}\\
\end{aligned}
\end{equation*}

\subsubsection*{b)}

\begin{equation*}
\begin{aligned}
& \text{Let } t = \frac{1}{c^Tx +d} \text{ \& } y = \frac{x}{c^Tx +d} \\
& g(y,t) = \frac{f_0(x)}{c^Tx + d}\\
& \text{$g_i$ is convex since the perspective of a convex function is convex}\\
& \text{For the constraints, the perspective still holds: }\\
& \text{Since $f_i$ is convex: } g_i(y,t) \le 0,\ i= 1,\dots,m\\
& \frac{x}{t} = y\\
& \frac{Ay}{t} = b\\
& Ay = bt\\
& t = \frac{1}{c^Tx + d}\\
& tc^T(\frac{y}{t}) + dt = 1\\
& c^Ty + dt = 1
\end{aligned}
\end{equation*}

\subsection*{4.8}

\subsubsection*{a)}

\begin{equation*}
\begin{aligned}
& \underset{x}{\text{minimize}}
& & c^Tx\\
& \text{subject to}
& & Ax = b\\
\end{aligned}
\end{equation*}
Three scenarios:\\
1)
\begin{equation*}
\begin{aligned}
& Ax = b \text{ is infeasible, then }\\
& p^* = \infty
\end{aligned}
\end{equation*}
2)
\begin{equation*}
\begin{aligned}
& Ax = b \text{ is feasible and $c\ \perp \ Null(A)$}\\
& p^* = c^Tx^* = c^Ty^*
\end{aligned}
\end{equation*}
3)
\begin{equation*}
\begin{aligned}
& Ax = b \text{ is feasible and $c\ \not\perp\ Null(A)$}\\
& \text{Problem is unbounded and: }\\
& p^* = -\infty
\end{aligned}
\end{equation*}
\subsubsection*{c)}
\begin{equation*}
\begin{aligned}
& \underset{x}{\text{minimize}}
& & c^Tx\\
& \text{subject to}
& & l \preceq 1 \preceq u \\
\end{aligned}
\end{equation*}
We multiple c into the constraint to get the values for $p^*$: 
\begin{equation*}
\begin{aligned}
c^Tl \preceq c^Tx \preceq c^Tu\\
p^* = 
    \begin{cases}
    c^Tl;& x = l, c > 0\\
    c^Tu;& x = u, c > 0\\
    c^Tx;& x \in [l,u], c = 0 
    \end{cases}
\end{aligned}
\end{equation*}


\subsection*{4.11}

\subsubsection*{b)}

\begin{equation*}
\begin{aligned}
& {\text{minimize}}
& & ||Ax-b||_1\\
\end{aligned}
\end{equation*}
\begin{equation*}
\begin{aligned}
& \sum_{i=1}^{n}|y_i-f(x_i)|\\
& = t\cdot1\\
& \text{For the norm given: }\\
& S = t\cdot1 = \sum_{1}^{n}|Ax-b|\\
\end{aligned}
\end{equation*}
Equivalent to the linear program:
\begin{equation*}
\begin{aligned}
& {\text{minimize}}
& & t\cdot1\\
& \text{subject to}
& & -t \preceq Ax-b \preceq t \\
\end{aligned}
\end{equation*}
The optimal solution is when the $k^{th}$ value is: $$|a_i^Tx-b|=t_i$$

\subsubsection*{b)}

\begin{equation*}
\begin{aligned}
& {\text{minimize}}
& & ||Ax-b||_1\\
& \text{subject to}
& & ||x||_{\infty} \le 1
\end{aligned}
\end{equation*}
This is equivalent to the linear program:
\begin{equation*}
\begin{aligned}
& {\text{minimize}}
& & t\cdot1\\
& \text{subject to}
& & -t \preceq Ax-b \preceq t \\
&&& ||x||_{\infty} \le 1
\end{aligned}
\end{equation*}
The last constraint is equivalent to:
\begin{equation*}
\begin{aligned}
& max(|x_1|,|x_2|,\dots,|x_n|) \le 1\\
& -\vec{1} \le x \le \vec{1}
\end{aligned}
\end{equation*}

\subsection*{4.12)}

We want to minimuze the given cost: 
$$C = \sum_{i,j=1}^{n}c_{ij}x_{ij}$$
We want to the net flow to be conserved at each node so that 
$$b_i + \sum_{j=1}^{n}x_{ij}-\sum_{j=1}^{n}x_{ji}=0,\ i=1,\dots ,n$$
Flow links are bounded as well: 
$$l_{ij}\le x_{ij} \le u_{ij}$$

\subsection*{4.15)}

\begin{equation*}
\begin{aligned}
& {\text{minimize}}
& & c^Tx\\
& \text{subject to}
& & Ax \preceq b\\
&&& x \in \set{0,1}, i=1,\dots,n
\end{aligned}
\end{equation*}
Relaxation method:
\begin{equation*}
\begin{aligned}
& {\text{minimize}}
& & c^Tx\\
& \text{subject to}
& & Ax \preceq b\\
&&& 0 \le x_i \le 1, i=1,\dots,n
\end{aligned}
\end{equation*}

\subsubsection*{a)}
The feasible set of the relaxed problem is a superset of the feasible set of the unrelaxed problem:
\begin{equation*}
\begin{aligned}
& \set{x | Ax \preceq b, x \in \set{0,1}, i=1,\dots,n} \supseteq \set{x | Ax \preceq b, 0 \le x_i \le 1, i=1,\dots,n}
\end{aligned}
\end{equation*}
This means that there exists a $x^*$ in relaxed feasible such that:
$$f(x^*_{relaxed})=p^*_{relaxed} \le f(x^*_{unrelaxed})=p^*_{unrelaxed}$$

\subsubsection*{b)}
If the solution to the relaxtion method is such that $x^* \in {0,1}$ then it is a solution to the boolean L.P.

\subsection*{4.23)}

\begin{equation*}
\begin{aligned}
& {\text{minimize}}
& & ||Ax-b||_4
\end{aligned}
\end{equation*}
Note that the norm is defined as:
$$
(\sum_{1}^{n}(a_i^Tx-b_1)^4)^{\frac{1}{4}}
$$
We solve via a change of variable metho to convert to a quadratic program:
\begin{equation*}
\begin{aligned}
& \text{Define: } y_i = a_i^Tx-b_i\\
& \text{Defintion} z_i = y_i^2
\end{aligned}
\end{equation*}
The new minimization problem is formulated as:
\begin{equation*}
\begin{aligned}
& {\text{minimize}}
& & \sum^{n}_{i=1}z^2\\
& \text{subject to}
& & y = Ax-b\\
&&& y^2 = z 
\end{aligned}
\end{equation*}\

\subsection*{4.40)}

\subsubsection*{c)}

\begin{equation*}
\begin{aligned}
& {\text{minimize}}
& & (Ax+b)^TF(x)^{-1}(Ax+b)
\end{aligned}
\end{equation*}
We defined this via an epigraph by definiing the above in terms of t-level sets:
The new minimization problem is formulated as:
$$
(Ax+b)^TF(x)^{-1}(Ax+b) \le t
$$
For which we minimize t.\\
The above is equivalent to the below block matrix through Shur's complement:
$$ 
\begin{bmatrix}
F(x) & Ax+b \\
(Ax+b)^T & t
\end{bmatrix} \succeq 0
$$
\begin{equation*}
\begin{aligned}
& {\text{minimize}}
& & t\\
& \text{subject to}
& & \begin{bmatrix}
F(x) & Ax+b \\
(Ax+b)^T & t
\end{bmatrix} \succeq 0
\end{aligned}
\end{equation*}

\subsection*{4.43)}

\subsubsection*{a)}

Using a property $\lamba_i(x) \le t$ for some $t$ iff $A(x) \preceq tI$.
\begin{equation*}
\begin{aligned}
& {\text{minimize}}
& & \lambda_1(x)\\
& \text{subject to}
& & A(x) \preceq tI
\end{aligned}
\end{equation*}

\subsubsection*{b)}
Using a property $\lamba_i(x) \le t$ for some $t$ iff $A(x) \preceq tI$ and
$\lamba_i(x) \ge \gamma$ for some $t$ iff $A(x) \succeq \lamba I$
\begin{equation*}
\begin{aligned}
& {\text{minimize}}
& & \lambda_1(x)\\
& \text{subject to}
& & \gamma I \preceq A(x) \preceq tI
\end{aligned}
\end{equation*}


% --------------------------------------------------------------
%     You don't have to mess with anything below this line.
% --------------------------------------------------------------
 
\end{document}